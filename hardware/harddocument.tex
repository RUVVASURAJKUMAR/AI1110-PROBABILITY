\let\negmedspace\undefined
\let\negthickspace\undefined
\documentclass[journal,12pt,twocolumn]{IEEEtran}
%\documentclass[conference]{IEEEtran}
%\IEEEoverridecommandlockouts
% The preceding line is only needed to identify funding in the first footnote. If that is unneeded, please comment it out.
\usepackage{cite}
\usepackage{amsmath,amssymb,amsfonts,amsthm}
\usepackage{algorithmic}
\usepackage{graphicx}
\usepackage{textcomp}
\usepackage{xcolor}
\usepackage{txfonts}
\usepackage{listings}
\usepackage{enumitem}
\usepackage{mathtools}
\usepackage{gensymb}
\usepackage[breaklinks=true]{hyperref}
\usepackage{tkz-euclide} % loads  TikZ and tkz-base
\usepackage{listings}
\usepackage{caption}
%
%\usepackage{setspace}
%\usepackage{gensymb}
%\doublespacing
%\singlespacing

%\usepackage{graphicx}
%\usepackage{amssymb}
%\usepackage{relsize}
%\usepackage[cmex10]{amsmath}
%\usepackage{amsthm}
%\interdisplaylinepenalty=2500
%\savesymbol{iint}
%\usepackage{txfonts}
%\restoresymbol{TXF}{iint}
%\usepackage{wasysym}
%\usepackage{amsthm}
%\usepackage{iithtlc}
%\usepackage{mathrsfs}
%\usepackage{txfonts}
%\usepackage{stfloats}
%\usepackage{bm}
%\usepackage{cite}
%\usepackage{cases}
%\usepackage{subfig}
%\usepackage{xtab}
%\usepackage{longtable}
%\usepackage{multirow}
%\usepackage{algorithm}
%\usepackage{algpseudocode}
%\usepackage{enumitem}
%\usepackage{mathtools}
%\usepackage{tikz}
%\usepackage{circuitikz}
%\usepackage{verbatim}
%\usepackage{tfrupee}
%\usepackage{stmaryrd}
%\usetkzobj{all}
%    \usepackage{color}                                            %%
%    \usepackage{array}                                            %%
%    \usepackage{longtable}                                        %%
%    \usepackage{calc}                                             %%
%    \usepackage{multirow}                                         %%
%    \usepackage{hhline}                                           %%
%    \usepackage{ifthen}                                           %%
  %optionally (for landscape tables embedded in another document): %%
%    \usepackage{lscape}     
%\usepackage{multicol}
%\usepackage{chngcntr}
%\usepackage{enumerate}

%\usepackage{wasysym}
%\newcounter{MYtempeqncnt}
\DeclareMathOperator*{\Res}{Res}
%\renewcommand{\baselinestretch}{2}
\renewcommand\thesection{\arabic{section}}
\renewcommand\thesubsection{\thesection.\arabic{subsection}}
\renewcommand\thesubsubsection{\thesubsection.\arabic{subsubsection}}

\renewcommand\thesectiondis{\arabic{section}}
\renewcommand\thesubsectiondis{\thesectiondis.\arabic{subsection}}
\renewcommand\thesubsubsectiondis{\thesubsectiondis.\arabic{subsubsection}}

% correct bad hyphenation here
\hyphenation{op-tical net-works semi-conduc-tor}
\def\inputGnumericTable{}                                 %%

\lstset{
%language=C,
frame=single, 
breaklines=true,
columns=fullflexible
}
%\lstset{
%language=tex,
%frame=single, 
%breaklines=true
%}

\begin{document}
%


\newtheorem{theorem}{Theorem}[section]
\newtheorem{problem}{Problem}
\newtheorem{proposition}{Proposition}[section]
\newtheorem{lemma}{Lemma}[section]
\newtheorem{corollary}[theorem]{Corollary}
\newtheorem{example}{Example}[section]
\newtheorem{definition}[problem]{Definition}
%\newtheorem{thm}{Theorem}[section] 
%\newtheorem{defn}[thm]{Definition}
%\newtheorem{algorithm}{Algorithm}[section]
%\newtheorem{cor}{Corollary}
\newcommand{\BEQA}{\begin{eqnarray}}
\newcommand{\EEQA}{\end{eqnarray}}
\newcommand{\define}{\stackrel{\triangle}{=}}

\bibliographystyle{IEEEtran}
%\bibliographystyle{ieeetr}


\providecommand{\mbf}{\mathbf}
\providecommand{\pr}[1]{\ensuremath{\Pr\left(#1\right)}}
\providecommand{\qfunc}[1]{\ensuremath{Q\left(#1\right)}}
\providecommand{\sbrak}[1]{\ensuremath{{}\left[#1\right]}}
\providecommand{\lsbrak}[1]{\ensuremath{{}\left[#1\right.}}
\providecommand{\rsbrak}[1]{\ensuremath{{}\left.#1\right]}}
\providecommand{\brak}[1]{\ensuremath{\left(#1\right)}}
\providecommand{\lbrak}[1]{\ensuremath{\left(#1\right.}}
\providecommand{\rbrak}[1]{\ensuremath{\left.#1\right)}}
\providecommand{\cbrak}[1]{\ensuremath{\left\{#1\right\}}}
\providecommand{\lcbrak}[1]{\ensuremath{\left\{#1\right.}}
\providecommand{\rcbrak}[1]{\ensuremath{\left.#1\right\}}}
\theoremstyle{remark}
\newtheorem{rem}{Remark}
\newcommand{\sgn}{\mathop{\mathrm{sgn}}}
\providecommand{\abs}[1]{\left\vert#1\right\vert}
\providecommand{\res}[1]{\Res\displaylimits_{#1}} 
\providecommand{\norm}[1]{\left\lVert#1\right\rVert}
%\providecommand{\norm}[1]{\lVert#1\rVert}
\providecommand{\mtx}[1]{\mathbf{#1}}
\providecommand{\mean}[1]{E\left[ #1 \right]}
\providecommand{\fourier}{\overset{\mathcal{F}}{ \rightleftharpoons}}
%\providecommand{\hilbert}{\overset{\mathcal{H}}{ \rightleftharpoons}}
\providecommand{\system}{\overset{\mathcal{H}}{ \longleftrightarrow}}
 %\newcommand{\solution}[2]{\textbf{Solution:}{#1}}
\newcommand{\solution}{\noindent \textbf{Solution: }}
\newcommand{\cosec}{\,\text{cosec}\,}
\providecommand{\dec}[2]{\ensuremath{\overset{#1}{\underset{#2}{\gtrless}}}}
\newcommand{\myvec}[1]{\ensuremath{\begin{pmatrix}#1\end{pmatrix}}}
\newcommand{\mydet}[1]{\ensuremath{\begin{vmatrix}#1\end{vmatrix}}}
%\numberwithin{equation}{section}
%\numberwithin{equation}{subsection}
%\numberwithin{problem}{section}
%\numberwithin{definition}{section}
%\makeatletter
%\@addtoreset{figure}{problem}
%\makeatother




\title{Hardware Assignment Report}
\author{AI22BTECH11022, RUVVA SURAJ KUMAR}
\date{\today}

\maketitle

\section{Description}

The Random number generator circuit operates with a 5V power supply obtained from a micro USB connection for power source like socket, laptop. This voltage is referred as Vcc which serves as the power source for the overall circuit.In my project The inner buses on both sides of the circuit are also connected to the Vcc, the lowest bus is connected to the ground(GND) and the uppermost bus carries the Clock signal generated by a 555 precision timer.

\section{Working of the circuit}

The circuit uses two D Flipflops, which receive the Clock signal from the clock bus. Depending on their initial state, the Flipflops generate a sequence of numbers as output. The generated sequence is predetermined and fixed. However, if the circuit is operated without considering the initial state of the Flipflops, the displayed numbers will appear to be randomly selected from a range of 0 to 15 (inclusive). Each number in this range has an equal probability of being displayed.

The circuit includes a 7447 Seven Segment Display Decoder, which is capable of displaying numbers from 0 to 15. Importantly, the values represented by the Flipflops (denoted as ABCD) never become 0000 at any point in time. Once all 16 numbers have been displayed, the output sequence repeats.

The circuit's behavior is deterministic, it means that the apparent randomness in the output can be decoded by simply referring to the predetermined sequence. The specific sequence generated by this circuit is as follows: 3, 7, 15, 14, 13, 10, 5, 11, 6, 12, 9, 2, 4, 8, 1, 3, 7, and so on.Refer, \refer{Fig:1} and \refer{Fig:2}




\section{Timer}

The time period of the circuit can be adjusted by using different combinations of resistors and capacitors. I used capacitors with values of 47nF and 470nF in this project. This configuration allows the circuit to produce a square pulse with a voltage of 5V approximately every 0.9 seconds. The slower pulse rate enables accurate readings to be taken from the resistor.

\begin{figure}[h]
    \centering
    \includegraphics[width = 0.5\textwidth]{pics/pic7.png}
    \caption{555 timer circuit}
    \label{fig:Fig:2}
\end{figure}

\begin{figure}[h]
    \centering
    \includegraphics[width = 0.5\textwidth]{pics/pic10.png}
    \caption{7447 to diplay connections}
    \label{fig:Fig:1}
\end{figure}   

\begin{figure}[h]
    \centering
    \includegraphics[width = 0.5\textwidth]{pics/pic9.png}
    \caption{7447}
    \label{fig:Fig:1}
\end{figure}       

\begin{figure}[h]
    \centering
    \includegraphics[width = 0.6\textwidth]{pics/pic3.png}
    \caption{display circuit}
    \label{fig:Fig:3}
\end{figure}



\section{Components Used}

\begin{enumerate}
  \item Breadboard: Used as the base platform for constructing the circuit.
  \item Seven Segment Display Common Anode: Displays the output numbers.
  \item 7447 Seven Segment Display Decoder: Converts the output numbers into signals that can be displayed on the seven-segment display.
  \item 7474 D FlipFlop x2: Flipflops responsible for generating the output number sequence based on the clock signal.
  \item 7486 XOR gate: Performs logical XOR operation.
  \item 555 precision timer: Generates the Clock signal used by the Flipflops.
  \item Resistor 10M$\Omega$: Used in the circuit for specific purposes.
  \item Resistor 1K$\Omega$: Used in the circuit for specific purposes.
  \item Capacitor 47nF: Substituted for one of the recommended capacitors in the timer circuit.
  \item Capacitor 470nF: Substituted for another recommended capacitor in the timer circuit.
  \item USB micro B breakout board: Provides the micro USB connection for the power supply.
  \item Jumper wires: Connect various components on the breadboard, enabling the flow of signals and power throughout the circuit.
\end{enumerate}


\begin{figure}[h]
    \begin{center}
    \includegraphics[width = 0.5\textwidth]{pics/pic4.jpg}
    \caption{My Project}
    \end{center}
    \label{fig:my_label} 
\end{figure}

\begin{figure}[h]
    \centering
    \includegraphics[width = 0.5\textwidth]{pics/pic8.jpg}
    \caption{My Project}
    \label{fig:my_label} 
\end{figure}

\begin{figure}[h]
    \centering
    \includegraphics[width = 1.0\textwidth]{pics/pic14.png}
    \caption{block chain diagram of the circuit}
    \label{fig:my_label} 
\end{figure}
         

\end{document}


